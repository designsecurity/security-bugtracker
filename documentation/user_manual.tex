\documentclass[a4paper]{report}

\usepackage[latin1]{inputenc} % accents
\usepackage[T1]{fontenc}      % caract�res fran�ais
\usepackage[frenchb, english]{babel}  % langue
\usepackage{graphicx}         % images
\usepackage{verbatim}         % texte pr�format�
\usepackage{listings}
\usepackage[]{algorithm2e}
\usepackage[margin=0.8in]{geometry}
\usepackage[table]{xcolor}
\usepackage{xcolor}
\usepackage[hidelinks]{hyperref}
\usepackage{color}

\definecolor{mygreen}{rgb}{0,0.6,0}
\definecolor{mygray}{rgb}{0.5,0.5,0.5}
\definecolor{mymauve}{rgb}{0.58,0,0.82}

\lstset{ %
  backgroundcolor=\color{white},   % choose the background color; you must add \usepackage{color} or \usepackage{xcolor}
  basicstyle=\footnotesize,        % the size of the fonts that are used for the code
  breakatwhitespace=false,         % sets if automatic breaks should only happen at whitespace
  breaklines=true,                 % sets automatic line breaking
  captionpos=b,                    % sets the caption-position to bottom
  commentstyle=\color{mygreen},    % comment style
  deletekeywords={...},            % if you want to delete keywords from the given language
  escapeinside={\%*}{*)},          % if you want to add LaTeX within your code
  extendedchars=true,              % lets you use non-ASCII characters; for 8-bits encodings only, does not work with UTF-8
  frame=single,	                   % adds a frame around the code
  keepspaces=true,                 % keeps spaces in text, useful for keeping indentation of code (possibly needs columns=flexible)
  keywordstyle=\color{white},       % keyword style
  language=Octave,                 % the language of the code
  otherkeywords={*,...},            % if you want to add more keywords to the set
  numbers=left,                    % where to put the line-numbers; possible values are (none, left, right)
  numbersep=5pt,                   % how far the line-numbers are from the code
  numberstyle=\tiny\color{mygray}, % the style that is used for the line-numbers
  rulecolor=\color{black},         % if not set, the frame-color may be changed on line-breaks within not-black text (e.g. comments (green here))
  showspaces=false,                % show spaces everywhere adding particular underscores; it overrides 'showstringspaces'
  showstringspaces=false,          % underline spaces within strings only
  showtabs=false,                  % show tabs within strings adding particular underscores
  stepnumber=2,                    % the step between two line-numbers. If it's 1, each line will be numbered
  stringstyle=\color{mymauve},     % string literal style
  tabsize=2,	                   % sets default tabsize to 2 spaces
  title=\lstname                   % show the filename of files included with \lstinputlisting; also try caption instead of title
}

\lstdefinestyle{customc}{
  belowcaptionskip=1\baselineskip,
  breaklines=true,
  frame=L,
  xleftmargin=\parindent,
  language=C,
  showstringspaces=false,
  basicstyle=\footnotesize\ttfamily,
  keywordstyle=\bfseries\color{green!40!black},
  commentstyle=\itshape\color{purple!40!black},
  identifierstyle=\color{blue},
  stringstyle=\color{orange},
}

\lstset{escapechar=@,style=customc}

\lstdefinestyle{DOS}
{
    backgroundcolor=\color{black},
    identifierstyle=\color{white},
    basicstyle=\scriptsize\color{white}\ttfamily
}

\addto\captionsfrancais{% Replace "english" with the language you use
  \renewcommand{\contentsname}%
    {Sommaire}%
}

\begin{document}

\begin{titlepage}
\fontfamily{phv}\selectfont
\vspace*{\stretch{1}}
\begin{flushright}\LARGE
Security bugtracker
\end{flushright}
\hrule
\begin{flushleft}\huge\bfseries
User Manual
\end
{flushleft}
\vspace*{\stretch{2}}
\begin{center}
Eric Therond
\end{center}
\end{titlepage}

\tableofcontents

\chapter{Installation}

  \section{Overview}

\textit{Security-bugtracker} is currently a tool based on three dependencies :
\begin{itemize}
\item webissues bug tracker : http://webissues.mimec.org/
\item openvas  : http://www.openvas.org/
\item dirb : http://dirb.sourceforge.net/
\item nmap : https://nmap.org/
\item arachni : www.arachni-scanner.com
\item sslscan : https://github.com/ssllabs/ssllabs-scan
\item node security platform : https://nodesecurity.io/ 
\item sensiolabs security checker : https://security.sensiolabs.org/
\item dependency check : https://github.com/jeremylong/DependencyCheck

\end{itemize}
\vspace{5mm}
Each of this tool can be installed on different or same server.\\
The aim of this project is to produce automated security tests and track detected default in a bugtracker.
\vspace{5mm}
\begin{center}
\includegraphics[scale=0.60]{sec1.png}
\vspace{5mm}
\end{center}
  \newpage
  
  \section{Openvas}
See the documentation on the official web site : http://www.openvas.org/install-source.html\\
On the same server install a web server and php, then copy the following module of this project to the directory of your web server :/security-bugtracker/security\_tools/openvas\\
Then edit /security-bugracker/security\_tools/openvas/openvas.conf.php :\\
\begin{lstlisting}[style=customc]
<?php

$CONF_WS_OPENVAS_LOGIN = "test";
$CONF_WS_OPENVAS_PASSWORD = "test";
$CONF_WEBISSUES_OPENVAS_LOGIN = "openvas";
$CONF_WEBISSUES_OPENVAS_PASSWORD = "openvas";
$CONF_WEBISSUES_WS_ENDPOINT = "http://localhost:8080/webissues-server-1.1.4/client/webservices.php";
$CONF_OPENVAS_ALERT_URL = "http://localhost:8080/webissues-server-1.1.4/client/security_tools/openvas/openvas.php";
$CONF_OPENVAS_ADMIN_LOGIN = "admin";
$CONF_OPENVAS_ADMIN_PASSWORD = "0825839c-0d3f-4417-a118-954a78e2553c";
$CONF_OPENVAS_CONFIG_ID = "a0e8fed8-45c1-4890-bd08-671257f63308";
$CONF_OPENVAS_PATH_OMP = "/usr/local/bin/omp";
$CONF_OPENVAS_PORT_OMP = "9393";

?>
\end{lstlisting}\
\begin{itemize}
\item \scriptsize{CONF\_WS\_OPENVAS\_LOGIN}
\item \scriptsize{CONF\_WS\_OPENVAS\_PASSWORD}
\end{itemize}
are the credentials for the web services of this module.
\begin{itemize}
\item \scriptsize{CONF\_WEBISSUES\_OPENVAS\_LOGIN}
\item \scriptsize{CONF\_WEBISSUES\_OPENVAS\_PASSWORD}
\item \scriptsize{CONF\_WEBISSUES\_WS\_ENDPOINT}
\end{itemize}
 will be completed later.
\begin{itemize}
\item \scriptsize{CONF\_OPENVAS\_ALERT\_URL}
\end{itemize}
 is the address of this module on this web server.
\begin{itemize}
\item \scriptsize{CONF\_OPENVAS\_ADMIN\_LOGIN}
\item \scriptsize{CONF\_OPENVAS\_ADMIN\_PASSWORD}
\end{itemize}
 are the openvas admin credentials.
\begin{itemize}
\item \scriptsize{CONF\_OPENVAS\_CONFIG\_ID}
\end{itemize}
is the default config id for run a scan with openvas, check your config with this openvas command
\begin{lstlisting}[style=DOS]
linux-3ig5:/home/eric/security-bugracker/documentation # omp -u admin -w 0825839c-0d3f-4417-a118-954a78e2553c -p 9393 --get-configs
8715c877-47a0-438d-98a3-27c7a6ab2196  Discovery
085569ce-73ed-11df-83c3-002264764cea  empty
daba56c8-73ec-11df-a475-002264764cea  Full and fast
698f691e-7489-11df-9d8c-002264764cea  Full and fast ultimate
708f25c4-7489-11df-8094-002264764cea  Full and very deep
a0e8fed8-45c1-4890-bd08-671257f63308  Full and very deep Clone 1
74db13d6-7489-11df-91b9-002264764cea  Full and very deep ultimate
2d3f051c-55ba-11e3-bf43-406186ea4fc5  Host Discovery
bbca7412-a950-11e3-9109-406186ea4fc5  System Discovery
\end{lstlisting}
\begin{itemize}
\item \scriptsize{CONF\_OPENVAS\_PATH\_OMP}
\end{itemize}
is the path of your omp binary on this server.
\begin{itemize}
\item \scriptsize{CONF\_OPENVAS\_PORT\_OMP}
\end{itemize}
is the tcp port which on openvas / omp is running
  \newpage

  \section{Dependency-check}
See the documentation on the official web site : https://github.com/jeremylong/DependencyCheck
  
  \section{Webissues}
See the documentation on the official web site : http://wiki.mimec.org/wiki/WebIssues/Installation. Once the bugtracker is installed, copy the following module of this project to your webissues root directory :\\
/security-bugracker/webissues-server-1.1.4\\\\
Next go at this address (replace the name, port, path with rights informations) :\\
http://localhost:8080/webissues-server-1.1.4/client/securityplugin.php
\newline
\includegraphics[scale=0.50]{sec2.png}
\vspace{5mm}
\newline
Select \textit{install plugin} and enter choosen values when the openvas module was installed above :\\
\begin{itemize}
\item \scriptsize{CONF\_WS\_OPENVAS\_LOGIN}
\item \scriptsize{CONF\_WS\_OPENVAS\_PASSWORD}
\item \scriptsize{CONF\_OPENVAS\_ALERT\_URL}
\end{itemize}
For finish create \textit{openvas} and \textit{Dependency-check} users in webissues.\\
Modify wsdl endpoints (openvas.wsdl and webservices.wsdl) to specify the correct address
\chapter{Use}
Don't forget to use \textit{basic authentification} with a login which have the good rights on webissues when using the webservices.
  \section{add a project}
Add a project with the following web service method or via the traditional him of web issues :
\begin{lstlisting}[style=customc]
<soapenv:Envelope xmlns:soapenv="http://schemas.xmlsoap.org/soap/envelope/" xmlns:v1="http://securitybugtracker/V1">
   <soapenv:Header/>
   <soapenv:Body>
      <v1:addproject>
         <name>TEST</name>
         <description>TEST</description>
      </v1:addproject>
   </soapenv:Body>
</soapenv:Envelope>
\end{lstlisting}
Remember the ids returned with the response :
\begin{lstlisting}[style=customc]
<SOAP-ENV:Envelope xmlns:SOAP-ENV="http://schemas.xmlsoap.org/soap/envelope/" xmlns:ns1="http://securitybugtracker/V1">
   <SOAP-ENV:Body>
      <ns1:addproject_Response>
         <id_details>
            <id_project>29</id_project>
            <id_folder_bugs>81</id_folder_bugs>
            <id_folder_servers>82</id_folder_servers>
            <id_folder_codes>83</id_folder_codes>
            <id_folder_scans>84</id_folder_scans>
         </id_details>
      </ns1:addproject_Response>
   </SOAP-ENV:Body>
</SOAP-ENV:Envelope>
\end{lstlisting}
\newpage
  \section{add a member}
  Add a \textit{robot} member for this project (the \textit{openvas} account created during the installation) :
\begin{lstlisting}[style=customc]
<soapenv:Envelope xmlns:soapenv="http://schemas.xmlsoap.org/soap/envelope/" xmlns:v1="http://securitybugtracker/V1">
   <soapenv:Header/>
   <soapenv:Body>
      <v1:addmember>
         <id_user>4</id_user>
         <id_project>29</id_project>
         <access>admin</access>
      </v1:addmember>
   </soapenv:Body>
</soapenv:Envelope>
\end{lstlisting}
\begin{lstlisting}[style=customc]
<SOAP-ENV:Envelope xmlns:SOAP-ENV="http://schemas.xmlsoap.org/soap/envelope/" xmlns:ns1="http://securitybugtracker/V1">
   <SOAP-ENV:Body>
      <ns1:addmember_Response>
         <result_details>
            <result>true</result>
         </result_details>
      </ns1:addmember_Response>
   </SOAP-ENV:Body>
</SOAP-ENV:Envelope>
\end{lstlisting}
  
  \section{add a server}
  Add a target server for this project, you can add multiple ips separated by the , character and the values of \textit{use parameter} must be one of thoses :
  
\begin{itemize}
\item Development : for a development environment server
\item Test : for a test environment server
\item Production : for a production environment server
\end{itemize}
\begin{lstlisting}[style=customc]
<soapenv:Envelope xmlns:soapenv="http://schemas.xmlsoap.org/soap/envelope/" xmlns:v1="http://securitybugtracker/V1">
   <soapenv:Header/>
   <soapenv:Body>
      <v1:addserver>
         <id_folder_servers>82</id_folder_servers>
         <hostname>eric-pc</hostname>
         <description>eric-pc</description>
         <use>Production</use>
         <ipsaddress>127.0.0.1</ipsaddress>
      </v1:addserver>
   </soapenv:Body>
</soapenv:Envelope>
\end{lstlisting}
\begin{lstlisting}[style=customc]
<SOAP-ENV:Envelope xmlns:SOAP-ENV="http://schemas.xmlsoap.org/soap/envelope/" xmlns:ns1="http://securitybugtracker/V1">
   <SOAP-ENV:Body>
      <ns1:addserver_Response>
         <result_addserver_details>
            <id_server>1676</id_server>
         </result_addserver_details>
      </ns1:addserver_Response>
   </SOAP-ENV:Body>
</SOAP-ENV:Envelope>
\end{lstlisting}
  
  \section{add a code}
  Add a target code path for this project, the \textit{code parameter} is the path of the directory which contain librairies to be scanned by the dependency-check security tool.
\begin{lstlisting}[style=customc]
<soapenv:Envelope xmlns:soapenv="http://schemas.xmlsoap.org/soap/envelope/" xmlns:v1="http://securitybugtracker/V1">
   <soapenv:Header/>
   <soapenv:Body>
      <v1:addcode>
         <id_folder_codes>83</id_folder_codes>
         <name>java test</name>
         <description>java tes</description>
         <code>/home/eric/test/libs-java</code>
      </v1:addcode>
   </soapenv:Body>
</soapenv:Envelope>
\end{lstlisting}
\begin{lstlisting}[style=customc]
<SOAP-ENV:Envelope xmlns:SOAP-ENV="http://schemas.xmlsoap.org/soap/envelope/" xmlns:ns1="http://securitybugtracker/V1">
   <SOAP-ENV:Body>
      <ns1:addcode_Response>
         <result_addcode_details>
            <id_code>1680</id_code>
         </result_addcode_details>
      </ns1:addcode_Response>
   </SOAP-ENV:Body>
</SOAP-ENV:Envelope>
\end{lstlisting}

  \section{scan the targets}
  \subsection{Dynamic scan with openvas}
Run a scan with openvas security tool, select openvas value for the \textit{tool parameter}, select a specific openvas config scan if you don't want to use the default config parametered during the installation and select a filter which can be :
\begin{itemize}
\item info : only add issues with a severity equal or upper to info
\item minor : only add issues with a severity equal or upper to minor
\item medium : only add issues with a severity equal or upper to medium
\item high : only add issues with a severity equal to high
\end{itemize}
\begin{lstlisting}[style=customc]
<soapenv:Envelope xmlns:soapenv="http://schemas.xmlsoap.org/soap/envelope/" xmlns:v1="http://securitybugtracker/V1">
   <soapenv:Header/>
   <soapenv:Body>
      <v1:addscan>
         <id_folder_scans>88</id_folder_scans>
         <name>test scan soap ui</name>
         <description>test scan soap ui</description>
         <tool>openvas</tool>
         <filter>medium</filter>
         <!--Optional:-->
         <id_config_openvas>?</id_config_openvas>
      </v1:addscan>
   </soapenv:Body>
</soapenv:Envelope>
\end{lstlisting}
\begin{lstlisting}[style=customc]
<SOAP-ENV:Envelope xmlns:SOAP-ENV="http://schemas.xmlsoap.org/soap/envelope/" xmlns:ns1="http://securitybugtracker/V1">
   <SOAP-ENV:Body>
      <ns1:addscan_Response>
         <result_addscan_details>
            <id_scan>2422</id_scan>
         </result_addscan_details>
      </ns1:addscan_Response>
   </SOAP-ENV:Body>
</SOAP-ENV:Envelope>
\end{lstlisting}

  \subsection{Static scan with dependency-check}
The static scan must me run localy, see jobs chapter.

  \section{Results}
You can view the results of your precedings actions with th him of webissues :
\newline
\includegraphics[scale=0.50]{sec4.png}
\vspace{5mm}
\newline
\includegraphics[scale=0.50]{sec5.png}
\vspace{5mm}
\newline
\includegraphics[scale=0.50]{sec6.png}
\vspace{5mm}
\newline
\includegraphics[scale=0.50]{sec7.png}
\vspace{5mm}
\newline
\chapter{Jobs}

You can easily script a job which can interact with your configuration management tool for example for requesting automatically the web services and running security scans.\\You can see examples in the jobs directory :\\
/security-bugracker/security\_tools/jobs/run\_dependencycheck.php\\
/security-bugracker/security\_tools/jobs/run\_openvas.php\\
% \listoffigures
% \listoftables

\end{document}
